\apendice{Plan de Calidad del Proyecto}

\section{Introducción}

El Plan de Calidad del Proyecto establece los procedimientos y estándares necesarios para asegurar que la implementación de PAM360 en NTT Data cumpla con los requisitos de calidad definidos. Este documento detalla las actividades de control de calidad que se llevarán a cabo durante el proyecto.

\section{Casos de uso}
A continuación, se describen los principales casos de uso identificados para la implementación de PAM360:

\begin{table}[p]
	\centering
	\begin{tabularx}{\linewidth}{ p{0.21\columnwidth} p{0.71\columnwidth} }
		\toprule
		\textbf{CU-1}    & \textbf{Instalación y Configuración de PAM360}\\
		\toprule
		\textbf{Versión}              & 1.0    \\
		\textbf{Autor}                & Alejandro Villar Solla \\
		\textbf{Requisitos asociados} & RF-01, RF-02 \\
		\textbf{Descripción}          & Este caso de uso describe los pasos para instalar y configurar PAM360 en el entorno virtualizado de NTT Data. \\
		\textbf{Precondición}         & 
		\begin{itemize}
			\item Entorno virtualizado preparado.
			\item Archivo de instalación de PAM360 disponible.
		\end{itemize}\\
		\textbf{Acciones}             &
		\begin{enumerate}
			\item Descargar el archivo de instalación de PAM360 desde el sitio web del proveedor.
			\item Copiar el archivo .exe a la máquina virtual utilizando MRemote.
			\item Ejecutar el instalador y seguir las instrucciones de instalación.
			\item Configurar los parámetros iniciales (puertos, credenciales administrativas).
		\end{enumerate}\\
		\textbf{Postcondición}        & 
		\begin{itemize}
			\item PAM360 instalado y corriendo en localhost:8282.
			\item Acceso a la interfaz web de PAM360.
		\end{itemize}\\
		\textbf{Excepciones}          & 
		\begin{itemize}
			\item Error durante la instalación debido a incompatibilidad de SO o errores comunes.
			\item Falta de permisos administrativos para ejecutar el instalador.
		\end{itemize}\\
		\textbf{Importancia}          & Alta \\
		\bottomrule
	\end{tabularx}
	\caption{CU-1 Instalación y Configuración de PAM360.}
\end{table}

\begin{table}[p]
	\centering
	\begin{tabularx}{\linewidth}{ p{0.21\columnwidth} p{0.71\columnwidth} }
		\toprule
		\textbf{CU-2}    & \textbf{Integración de PAM360 con Active Directory}\\
		\toprule
		\textbf{Versión}              & 1.0    \\
		\textbf{Autor}                & Alejandro Villar Solla \\
		\textbf{Requisitos asociados} & RF-03, RF-04 \\
		\textbf{Descripción}          & Este caso de uso describe el proceso de integración de PAM360 con el Active Directory (AD) de NTT Data para sincronización de usuarios y roles. \\
		\textbf{Precondición}         & 
		\begin{itemize}
			\item PAM360 instalado y configurado.
			\item Acceso administrativo a Active Directory.
		\end{itemize}\\
		\textbf{Acciones}             &
		\begin{enumerate}
			\item Acceder a la interfaz web de PAM360.
			\item Navegar a la configuración de administración y seleccionar "Import From Active Directory".
			\item Ingresar el nombre del dominio AD y las credenciales de la cuenta asociada.
			\item Seleccionar los usuarios a sincronizar.
			\item Confirmar y ejecutar la integración.
		\end{enumerate}\\
		\textbf{Postcondición}        & 
		\begin{itemize}
			\item Usuarios y roles del AD sincronizados en PAM360.
			\item Accesos privilegiados gestionados centralmente desde PAM360.
		\end{itemize}\\
		\textbf{Excepciones}          & 
		\begin{itemize}
			\item Error de autenticación con AD.
		\end{itemize}\\
		\textbf{Importancia}          & Alta \\
		\bottomrule
	\end{tabularx}
	\caption{CU-2 Integración de PAM360 con Active Directory.}
\end{table}

\begin{table}[p]
	\centering
	\begin{tabularx}{\linewidth}{ p{0.21\columnwidth} p{0.71\columnwidth} }
		\toprule
		\textbf{CU-3}    & \textbf{Configuración de Políticas de Seguridad en PAM360}\\
		\toprule
		\textbf{Versión}              & 1.0    \\
		\textbf{Autor}                & Alejandro Villar Solla \\
		\textbf{Requisitos asociados} & RF-05, RF-06 \\
		\textbf{Descripción}          & Este caso de uso describe el procedimiento para configurar las políticas de seguridad dentro de PAM360, incluyendo autenticación y gestión de contraseñas. \\
		\textbf{Precondición}         & 
		\begin{itemize}
			\item PAM360 instalado y configurado.
			\item Acceso administrativo a la consola de PAM360.
		\end{itemize}\\
		\textbf{Acciones}             &
		\begin{enumerate}
			\item Iniciar sesión en la consola web de PAM360.
			\item Navegar a la sección de configuración de administración.
			\item Seleccionar "Políticas de Seguridad" y luego "Gestión de Contraseñas".
			\item Configurar las reglas de complejidad de contraseñas, frecuencia de cambio y políticas de bloqueo de cuentas.
			\item Guardar y aplicar las políticas configuradas.
		\end{enumerate}\\
		\textbf{Postcondición}        & 
		\begin{itemize}
			\item Políticas de seguridad y contraseñas aplicadas y activas en PAM360.
			\item Refuerzo de la seguridad en la gestión de accesos privilegiados.
		\end{itemize}\\
		\textbf{Excepciones}          & 
		\begin{itemize}
			\item Error al guardar las políticas debido a conflictos con configuraciones existentes.
			\item Usuarios no cumplen con las nuevas políticas de contraseña y necesitan soporte.
		\end{itemize}\\
		\textbf{Importancia}          & Alta \\
		\bottomrule
	\end{tabularx}
	\caption{CU-3 Configuración de Políticas de Seguridad en PAM360.}
\end{table}

\section{Objetivos de Calidad}

Los objetivos de calidad del proyecto incluyen:

\begin{itemize}
	\item Garantizar que PAM360 se instale y configure correctamente según las especificaciones del proveedor y los requisitos de NTT Data.
	\item Asegurar que todas las funcionalidades de PAM360 se integren y operen sin problemas con el Active Directory de NTT Data.
	\item Cumplir con los estándares de seguridad y buenas prácticas en la gestión de accesos privilegiados.
	\item Obtener retroalimentación y aprobación de los usuarios finales y administradores del sistema.
\end{itemize}

\section{Estrategia de Calidad}

Para alcanzar los objetivos de calidad, se adoptarán las siguientes estrategias:

\begin{itemize}
	\item \textbf{Revisión y Validación de Requisitos}: Verificar que todos los requisitos del proyecto estén claros, completos y sean factibles.
	\item \textbf{Pruebas Exhaustivas}: Realizar pruebas de instalación, configuración, integración y funcionalidad para asegurar que el sistema opere correctamente.
	\item \textbf{Auditorías de Seguridad}: Ejecutar auditorías para evaluar la seguridad del sistema y la conformidad con las políticas de seguridad de NTT Data.
	\item \textbf{Capacitación de Usuarios}: Asegurar que los usuarios finales y administradores reciban formación adecuada para utilizar y gestionar PAM360 eficientemente.
\end{itemize}

\section{Procesos para Garantizar la Calidad}

\subsection{Revisión de Documentación}

\begin{itemize}
	\item \textbf{Documentación Técnica}: Revisión de manuales, guías de instalación y configuración proporcionados por ManageEngine.
\end{itemize}

\subsection{Pruebas y Validaciones}

\begin{itemize}
	\item \textbf{Pruebas de Instalación}: Verificación de la correcta instalación de PAM360 en el entorno virtualizado.
	\item \textbf{Pruebas de Configuración}: Asegurar que todas las configuraciones necesarias (autenticación, políticas de seguridad, integración con AD) se realicen correctamente.
	\item \textbf{Pruebas Funcionales}: Validación de todas las funcionalidades de PAM360, incluyendo la gestión de usuarios, contraseñas y auditorías.
	\item \textbf{Pruebas de Seguridad}: Realizar pruebas de penetración y vulnerabilidad para identificar y mitigar posibles riesgos de seguridad.
\end{itemize}

\subsection{Control de Cambios}

\begin{itemize}
	\item \textbf{Gestión de Cambios}: Implementar un sistema de control de cambios para manejar las modificaciones en los requisitos, el diseño o la configuración de PAM360.
	\item \textbf{Aprobación de Cambios}: Todos los cambios deben ser revisados y aprobados por el equipo de proyecto antes de su implementación.
\end{itemize}

\section{Criterios de Aceptación}

Los criterios para aceptar el proyecto incluyen:

\begin{itemize}
	\item \textbf{Cumplimiento de Requisitos}: Todos los requisitos del proyecto deben ser cumplidos satisfactoriamente.
	\item \textbf{Validación de Funcionalidades}: Todas las funcionalidades de PAM360 deben operar sin errores y conforme a las especificaciones.
	\item \textbf{Resultados de Pruebas y Auditorías}: Los resultados de las pruebas y auditorías deben demostrar que el sistema es seguro y funcional.
\end{itemize}

\section{Métricas de Calidad}

Para medir la calidad del proyecto, se utilizarán las siguientes métricas:

\begin{itemize}
	\item \textbf{Tasa de Éxito de Pruebas}: Porcentaje de pruebas que se completan exitosamente sin errores.
	\item \textbf{Número de Incidentes de Seguridad}: Cantidad de incidentes de seguridad detectados y corregidos durante el proyecto.
	\item \textbf{Satisfacción de Usuarios}: Nivel de satisfacción de los usuarios finales y administradores con el sistema implementado.
	\item \textbf{Tiempo de Respuesta a Incidentes}: Tiempo promedio para detectar y resolver incidentes o problemas en el sistema.
\end{itemize}

\section{Responsabilidades}

\begin{itemize}
	\item \textbf{Gestor de Calidad}: Responsable de la planificación, ejecución y seguimiento de las actividades de aseguramiento y control de calidad.
	\item \textbf{Equipo de Proyecto}: Participar en las pruebas, auditorías y revisiones de calidad.
\end{itemize}

